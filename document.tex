\documentclass[12pt]{article}
\usepackage{amsmath}
\usepackage{graphicx}

\title{Blackboard Application}
\author{Fatimah Al-Jedaani (444014738) \\ Shahad Ali (444009300)}
\date{}

\begin{document}
	
	\begin{titlepage}
		\centering
		\vspace*{1cm}
		
		\textbf{\large{Kingdom of Saudi Arabia}} \\
		\textbf{\large{Ministry of Education}} \\
		\textbf{\large{Umm Al-Qura University}} \\
		\textbf{\large{College of Engineering and Computers}} \\
		\textbf{\large{Department of Computer Science}} \\
		
		\vfill
		
    \textbf{\Huge{Blackboard Application}} \\
	
		
		\vfill
		
		\large
		\textbf{Student name:} \\
		Fatimah Al-Jedaani (444014738) \\
		Shahad Ali (444009300)
		
		\vspace{1cm}
		
		\vfill
		
	\end{titlepage}
	
	\newpage
	
	\section*{          
		      With the rapid development of information technology, Learning Management Systems (LMS) have become an essential part of enhancing the educational process. Among these systems, the Blackboard application is one of the most prominent educational tools widely used in universities and educational institutions around the world. This application allows students and faculty members to access educational content, submit assignments, participate in discussions, and manage various aspects of the academic process smoothly and effectively. 
		             In this report, we will review the various types of key features and characteristics of the Blackboard application, and how it contributes to improving the teaching and learning experience, focusing on the benefits it provides to both teachers and students.
	}
	
	
	
	\newpage
	
	\section*{Report Content}
	
	\begin{enumerate}
		\item Introduction \dotfill 1
		\item The reason for the existence of Blackboard \dotfill 4
		\item Problems Resolved by Blackboard \dotfill 4
		\item Questionnaire \dotfill 5
		\item Comparison between Blackboard and Madrasty \dotfill 7
		\item Conclusion \dotfill 10
		\item References \dotfill 10
	\end{enumerate}
	
	\newpage
		\begin{titlepage}		
				\centering
			\vspace*{1cm}
	\vfill

\textbf{\Huge{Chapter 1}} \\


\vfill

\large
		\end{titlepage}
	
	\section{Introduction}
\subsection*{\textbf{Blackboard Learn provides an easy way to interact with different types of features, depending on your role, either as an instructor or a student.}	}
	
	\subsection{Introduction}             Blackboard is one of the modern educational applications in the modern era, as it provides a comprehensive platform for e-learning and manages educational content online. This program is used in many educational institutions around the world, ranging from multi-access educational institutions to higher education institutions.
	
	\subsection{Purpose}
	    The primary purpose of Blackboard is to facilitate content management, delivery, and assessment, featuring a variety of educational tools for seamless access to course materials, assignments, discussions, and assessments.
	
	\subsection{Stakeholders}
	\begin{itemize}
		\item User (Student - Teacher)
		\item Distance learning team
	\end{itemize}
	
	\subsection{Main tasks of Blackboard}
	\begin{enumerate}
		\item  Blackboard allows instructors to create and manage courses. This  	     includes organizing and uploading course materials such as syllabi, lecture notes, and multimedia content. 
		\item It provides a variety of communication tools such as discussion boards, messaging systems, and announcements.
		\item Blackboard supports various assessment types, including quizzes, tests, and assignments. It provides automatic grading for some assessments and tools for manual grading and feedback.
	\end{enumerate}
	
	\subsection{Requirements}
	\begin{enumerate}
		\item The system shall be easy to use and understandable for users regardless of their technological experience.
		\item User data must be well protected and secured.
		\item The system shall respond quickly.
		\item The system shall be available 24 hours.
	\end{enumerate}
	
	\newpage
		\begin{titlepage}		
		\centering
		\vspace*{1cm}
		\vfill
		
		\textbf{\Huge{Chapter2  } } \\
		
		
		\vfill
		
		\large
	\end{titlepage}
\section{Survey and Identified Problems}

  A survey was conducted to collect opinions about Blackboard and some problems were identified such as difficulty in dealing with the system and some technical problems such as difficulty in submitting assignments, which we examined through the survey and which most students suffer from. Among the solutions to this problem is making improvements in building the application, which will improve the strength of the application's performance and increase the speed of receiving assignments. Also, the difficulty of communication methods between teachers and students due to the lack of realistic conversations to communicate between them. Therefore, we decided to improve communication by creating a chat interface that allows communication between teachers and students in the application.
  
  
  Explains the differences between blackboard and madrasty in terms of advantages and disadvantages.
  

\section{Comparison between Blackboard and Madrasty}

Blackboard and Madrasty are both popular educational platforms designed to enhance teaching and learning experiences. However, they offer distinct features and cater to different needs. Lets compare their advantages and disadvantages:

\subsection{Blackboard}

\textbf{Advantages:}
\begin{itemize}
\item -	Widely used and established: Blackboard has been a staple in many educational institutions for years, making it a familiar platform for both students and faculty.
\item -	It offers a comprehensive suite of tools, including course management, online testing, discussion forums, and collaboration features.
\item-	Blackboard can integrate with other educational tools and systems, such as learning management systems (LMS) and student information systems (SIS).

\end{itemize}


\newpage
\textbf{Disadvantages:}
\begin{itemize}
\item -	The interface can be complex for new users due to its extensive features.
\item -	Like any large-scale platform, Blackboard can encounter technical issues

\end{itemize}

\subsection{Madrasty}

\textbf{Advantages:}
\begin{itemize}
\item
 -	Madrasty has a user-friendly interface, designed to be intuitive and easy to navigate, making it accessible to users of all ages.
\item
-	The app is optimized for mobile devices, allowing students to access their courses and assignments on the go.
\item
-	Madrasty is specifically tailored for Arabic-speaking users, providing features and content relevant to their needs.

\end{itemize}

\textbf{Disadvantages:}
\begin{itemize}
\item 
-	It may not offer the same breadth of features like Blackboard.
\item 
-	Madrasty might have a smaller user community and fewer resources available.

\end{itemize}




\begin{table}[ht]
	\subsection*{The following table is a comparison between Blackboard and  madrasty  in terms of security, platform, and   management}
	\centering
	\begin{tabular}{|c|c|c|}
		\hline
		& \textbf{Blackboard} & \textbf{Madrasty} \\ 
		\hline
		\textbf{Security} & 
		\begin{tabular}[c]{@{}l@{}} 
			- Both employ security measures to protect user data. \\ 
			- Use secure login and authentication processes. \\
			- Regularly update systems to address vulnerabilities. \\
			- Comply with relevant data protection regulations. \\
		\end{tabular} & 
		\begin{tabular}[c]{@{}l@{}} 
			- Both employ security measures to protect user data. \\ 
			- Use secure login and authentication processes. \\
			- Regularly update systems to address vulnerabilities. \\
			- Comply with relevant data protection regulations. \\
		\end{tabular} \\ 
		\hline
		\textbf{Platform} & 
		\begin{tabular}[c]{@{}l@{}} 
			- Web-based, mobile apps, integrates with \\ third-party tools, comprehensive LMS. \\ 
		\end{tabular} & 
		\begin{tabular}[c]{@{}l@{}} 
			- Primarily mobile app, web version available, \\ focuses on Middle East and North Africa. \\
		\end{tabular} \\ 
		\hline
		\textbf{Management} & 
		\begin{tabular}[c]{@{}l@{}} 
			- Both provide centralized systems for managing courses, \\ users, content, and assessments. \\ 
			- Allow administrators to manage user roles and access control. \\
			- Offer tools for tracking student progress and performance. \\
		\end{tabular} & 
		\begin{tabular}[c]{@{}l@{}} 
			- Both provide centralized systems for managing courses, \\ users, content, and assessments. \\ 
			- Allow administrators to manage user roles and access control. \\
			- Offer tools for tracking student progress and performance. \\
		\end{tabular} \\ 
		\hline
	\end{tabular}
	\caption{Comparison between Blackboard and Madrasty in terms of security, platform, and management.}
\end{table}



	\section*{  \textbf{  Blackboard is the leading learning management system (LMS), providing a powerful platform for institutions to create, manage, and deliver online courses that foster an interactive and engaging learning experience. This assignment will show you why Blackboard exists, and introduce some of the features and problems it solves.}}
	
	\subsection{The reason for the existence of Blackboard}
Blackboards creation was driven by the need for a more efficient and effective way to deliver education. Traditional classroom settings were often limited by geographical constraints, time constraints, and the availability of resources. Blackboard aimed to address these limitations by providing a flexible and scalable platform for online learning.
	
	\subsection{Problems Resolved by Blackboard}

	     	•\textbf{	Accessibility: }Blackboard has made education more 
	accessible  students who live in remote areas or have disabilities        
	
	•\textbf{	Flexibility}: It offers students the flexibility to learn at their own pace and on their own schedule 
	
	•\textbf{	Cost-Effectiveness}: By reducing the need for physical classrooms and materials Blackboard can help institutions save costs
	
	•	\textbf{Collaboration}: Blackboard fosters collaboration among students and instructors, promoting a more interactive and engaging learning experience 

		\begin{titlepage}
	\subsection{Questionnaire}
	
	
	\textbf{\section*{	conducted a survey to collect opinions about the blackboard application:}}
	Figure 1 : shows details to describe the interface used and the speed of this system.
	\includegraphics{صورة1}
	
	\includegraphics{صورة2}
	
	
	Figure 2 : shows the difficulty level of entering the lecture and communicating with teachers.
	
	\includegraphics{صورة3}
	
	\includegraphics{صورة4}
	
	
	
	
	\includegraphics{صورة5}
	
	Figure 3 : illustrates the additional features and problems of using Blackboard.
		\end{titlepage}
	A survey was conducted to collect opinions about the Blackboard application. Some problems were identified, including difficulties with the system, technical issues, and inconsistent notifications. Based on the survey, the following improvements were suggested:
	\begin{itemize}
		\item Improve the user interface and the speed of the program.
		\item Facilitate communication between teachers and students.
		\item Modify and add features to enhance the user experience.
	\end{itemize}
	
	\	\begin{titlepage}		
	\centering
	\vspace*{1cm}
	\vfill
	
	\textbf{\Huge{Chapter 3}} \\
	
	
	\vfill
	
	\large
	\end{titlepage}
	
	\section{Chapter 3}
	
	The Blackboard application database includes user data from different institutions. It manages and stores data related to students, teachers, assessments, lectures, and tests using relational databases like Microsoft SQL Server and MySQL. 
	
	\subsection{Programming Languages and Technologies}
	Blackboard relies on a variety of technologies:
	\begin{itemize}
		\item \textbf{JavaScript:} Used for front-end interactions and dynamic interfaces.
		\item \textbf{PHP:} Used in legacy applications and certain services.
		\item \textbf{SQL:} For database queries and management.
	\end{itemize}
	
	
	
	
	
	
	
	
	
	
	
	
	
	
	
	
	
	
	
		\newpage\
		
	
	\section{Conclusion}
	Blackboard has transformed education by addressing the limitations of traditional systems. With a comprehensive set of features, it is poised to continue shaping the future of learning.
	
	\section{References}
	\begin{itemize}
		\item Blackboard Inc. (n.d.). Blackboard. Retrieved from \texttt{https://www.blackboard.com/}
		\item Educational Technology Solutions. (n.d.). Blackboard: A comprehensive guide. Retrieved from \texttt{https://www.blackboard.com/}
		\item University of Michigan. (n.d.). Blackboard help. Retrieved from \texttt{https://dataguide.umflint.edu/services/view/13}
		\item The Chronicle of Higher Education. (2023). Blackboard's new CEO aims to reimagine higher education. Retrieved from \texttt{https://www.chronicle.com/}
	\end{itemize}
	
\end{document}

\begin{tabular}
	\hline
	\textbf{Feature} & \textbf{Blackboard} & \textbf{Madrasty} \\ \hline
	\textbf{Security} & Both use secure login and authentication, regularly updated, compliant with data protection regulations & Same as Blackboard \\ \hline
	\textbf{Platform} & Web-based, mobile apps, integrates with third-party tools, comprehensive LMS & Primarily mobile app, web version available, focuses on MENA region \\ \hline
	\textbf{Management} & Centralized system for managing courses, users, and content & Centralized system, tailored for MENA region \\ \hline
\end{tabular}