\documentclass[a4paper,12pt]{article}
\usepackage{graphicx}
\usepackage{hyperref}
\usepackage{longtable}
\usepackage{url}
\usepackage{breakurl}
\usepackage{booktabs} % لتحسين تنسيق الجدول
\usepackage{adjustbox} % لضبط حجم الجدول


\title{Blackboard Application}

\date{}

\begin{document}

\noindent
\begin{minipage}{0.7\textwidth}
    \raggedright
    \textbf{Kingdom of Saudi Arabia}\\
    \textbf{Ministry of Education}\\
    \textbf{Umm Al-Qura University}\\
    \textbf{College of Engineering and Computers}\\
 \textbf{   (Department of Computer Science)\\}
\end{minipage}%
\begin{minipage}{0.3\textwidth}
    \raggedleft
    \includegraphics[width=2cm]{logo.png} % Replace 'logo.png' with the path to your logo
\end{minipage}

\vspace{5cm}

% Centered Title
\begin{center}
    \LARGE \textbf{Blackboard Application}
\end{center}


\begin{center}
Fatimah Al-Jedaani (444014738)\\
Shahad Ali (444009300)\\
\textbf{\the\year}
\end{center}

\newpage

\begin{center}
    \LARGE \textbf{Overview}
\end{center}
With the rapid development of information technology, Learning
Management Systems (LMS) have become an essential part of
enhancing the educational process. Among these systems, the
Blackboard application is one of the most prominent educational tools
widely used in universities and educational institutions around the
world. This application allows students and faculty members to
access educational content, submit assignments, participate in
discussions, and manage various aspects of the academic process
smoothly and effectively.\newline

 In this report, we will review the various types of key features
and characteristics of the Blackboard application, and how it
contributes to improving the teaching and learning experience,
focusing on the benefits it provides to both teachers and students. 


\newpage

\tableofcontents{}

\newpage 

\begin{center}
\vspace*{10cm} 
\section{Chapter 1}
\end{center}



\newpage
Blackboard Learn provides an easy way to interact with different types of features,
depending on your role, either as an instructor or a student.
\subsection{Introduction}
Blackboard is one of the modern educational applications in the modern era, as
it provides a comprehensive platform for e-learning and manages educational
content online. This program is used in many educational institutions around the world, ranging from multi-access educational institutions to higher
education institutions.


\subsection{Purpose}
The primary purpose of Blackboard is to facilitate content management, delivery,and assessment, featuring a variety of educational tools for seamless access to course materials, assignments, discussions, and assessments.

\subsection{Stakeholders}
\begin{itemize}
    \item \textbf{Users:} Students and Teachers
    \item \textbf{Distance Learning Team:} Administrators responsible for managing online learning.
\end{itemize}


\subsection{Main Tasks of Blackboard}
\begin{itemize}
    \item Blackboard allows instructors to create and manage courses. This includes organizing and uploading course materials such as syllabi, lecture notes, and multimedia content. 
    \item Blackboard provides a variety of communication tools. This includes
discussion boards, messaging systems, and announcements. 
    \item Blackboard supports a range of assessment types, including quizzes, tests,and assignments. also provides automatic grading for certain assessments and provides tools for manual grading and feedback.
\end{itemize}

\newpage

\subsection{Functional Requirements}

\subsubsection{User Requirements}
\begin{itemize}
    \item The system must allow the user to log in using the username and password and manage his profile.
    \item The system must support sending notifications and alerts to the user to inform him of status updates.
    \item The user's system must be accessible via multiple devices.
    \item The system must provide the user with the ability to access files and courses.
\end{itemize}


\subsubsection{System Requirements}
\begin{itemize}
    \item The user shall log in to the system with the user's name and password.
    \item The system must support sending notifications and alerts to users to inform them of status updates. 
    \item The system must be accessible via multiple devices, ensuring cross-platform compatibility.

    \item - The system must provide users with the ability to access files and courses easily.
\end{itemize}

\subsection{Non-Functional Requirements}
\begin{itemize}
    \item The system shall be easy to use and understandable for users regardless of their technological experiences.
    \item User data must be well protected and secured.
    \item The system shall be Quick response.
    \item The system shall be available 24 hours.
\end{itemize}

\subsection{Conclusion}
In conclusion, the requirements and purpose of Blackboard have been outlined.
The Blackboard application is an essential tool that enhances the effectiveness of modern education by combining flexibility and ease of use. Its many features make adopting this technology a step towards achieving an integrated and distinguished educational experience.


\newpage 

\begin{center}
\vspace*{10cm} 
\section{Chapter 2}
\end{center}

\newpage


Blackboard is the leading learning management system (LMS), providing a
powerful platform for institutions to create, manage, and deliver online courses that foster an interactive and engaging learning experience. This assignment will show you why Blackboard exists, and introduce some of the features and problems it solves.

\subsection{The Reason for the Existence of Blackboard}
Blackboards creation was driven by the need for a more efficient and
effective way to deliver education. Traditional classroom settings were often
limited by geographical constraints, time constraints, and the availability of
resources. Blackboard aimed to address these limitations by providing a flexible and scalable platform for online learning.

\subsection{Problems Resolved by Blackboard}
\begin{itemize}
    \item \textbf{Accessibility:} Blackboard has made education more
 accessible students who live in remote areas or have disabilities.
    \item \textbf{Flexibility:}  It offers students the flexibility to learn at their own pace and on their own schedule.
    \item \textbf{Cost-Effectiveness:}  By reducing the need for physical classrooms and materials Blackboard can help institutions save costs.
    \item \textbf{Collaboration:} : Blackboard fosters collaboration among students and instructors, promoting a more interactive and engaging learning experience
\end{itemize}

\newpage

\subsection{Questionnaire}
conducted a survey to collect opinions about the blackboard application:
 \url{https://form.responsly.com/f/egEOjdNk}. 

\begin{figure}[h]
    \centering
    \includegraphics[width=0.55\textwidth]{Figure1.png} 
    \caption{Details describing the interface and speed of the Blackboard system.}
    \label{fig:figure1}
\end{figure}


\begin{figure}[h]
    \centering
    \includegraphics[width=0.55\textwidth]{figure2.png} 
    \caption{Difficulty level of entering lectures and communicating with teachers.}
    \label{fig:figure2}
\end{figure}

\newpage

\begin{figure}[h]
    \centering
    \includegraphics[width=0.7\textwidth]{figure3.png}
    \caption{Additional features and problems of using Blackboard.}
    \label{fig:figure3}
\end{figure}

A survey was conducted to gather opinions about Blackboard, as shown in
Figure \ref{fig:figure1}. identifying several issues that most students experience, such as difficulty in dealing with the system and technical problems with assignment submissions.
which were explored through the survey, shown in Figure \ref{fig:figure2} and Figure \ref{fig:figure3}. Among the solutions to this problem is making improvements in building the application, which will improve the strength of the application's performance and increase the speed of receiving assignments. Also, the difficulty of communication methods between teachers and students due to the lack of realistic conversations to communicate between them. Therefore, we decided to improve communication by creating a chat interface that allows communication between teachers and students in the
application.

\newpage

\subsection{Comparison between Blackboard and Madrasty}
Blackboard and Madrasty are both popular educational platforms designed to
enhance teaching and learning experiences. However, they offer distinct features and cater to different needs. Lets compare their advantages and disadvantages:

\textbf{Blackboard}

\textbf{Advantages:}
\begin{itemize}
    \item Widely used and established: Blackboard has been a staple in many
educational institutions for years, making it a familiar platform for both
students and faculty.
    \item It offers a comprehensive suite of tools, including course management,
online testing, discussion forums, and collaboration features.
    \item Blackboard can integrate with other educational tools and systems, such as
learning management systems (LMS) and student information systems
(SIS).
\end{itemize}

\textbf{Disadvantages:}
\begin{itemize}
    \item The interface can be complex for new users due to its extensive features.
    \item Like any large-scale platform, Blackboard can encounter technical issues.
\end{itemize}
\newpage
\textbf{Madrasty}

\textbf{Advantages:}
\begin{itemize}
    \item Madrasty has a user-friendly interface, designed to be intuitive and easy to
navigate, making it accessible to users of all ages.
    \item The app is optimized for mobile devices, allowing students to access their
courses and assignments on the go.
    \item Madrasty is specifically tailored for Arabic-speaking users, providing
features and content relevant to their needs.
\end{itemize}

\textbf{Disadvantages:}
\begin{itemize}
    \item It may not offer the same breadth of features like Blackboard.

    \item Madrasty might have a smaller user community and fewer resources
available.
\end{itemize}
\newpage
The following table is a comparison between Blackboard and madrasty in terms of security,platform, and management.\newline



\begin{table}[h]

    \centering
    \begin{adjustbox}{max width=\textwidth} % ضبط حجم الجدول
    \begin{tabular}{@{}lll@{}}
        \toprule
        \textbf{Feature} & \textbf{Blackboard} & \textbf{Madrasty} \\ \midrule
        \textbf{Security} & 
        \begin{tabular}[c]{@{}l@{}}
            - Both employ security measures to protect user data. \\
            - Use secure login and authentication processes. \\
            - Regularly update systems to address vulnerabilities. \\
            - Comply with relevant data protection regulations. \\
        \end{tabular} &
        \begin{tabular}[c]{@{}l@{}}
            - Both employ security measures to protect user data. \\
            - Use secure login and authentication processes. \\
            - Regularly update systems to address vulnerabilities. \\
            - Comply with relevant data protection regulations. \\
        \end{tabular} \\ \midrule
        
        \textbf{Platform} & 
        \begin{tabular}[c]{@{}l@{}}
            Web-based, mobile apps, integrates with third-party tools, \\comprehensive LMS\\
        \end{tabular} &
        \begin{tabular}[c]{@{}l@{}}
            Primarily mobile app, web version available, \\
            focuses on Middle East and North Africa\\
        \end{tabular} \\ \midrule
        \textbf{Management} & 
        \begin{tabular}[c]{@{}l@{}}
            - Both provide centralized systems for managing courses, users, \\ 
            content, and assessments. \\
            - Allow administrators to manage user roles and access control. \\
            - Offer tools for tracking student progress and performance. \\
        \end{tabular} &
        \begin{tabular}[c]{@{}l@{}}
            - Both provide centralized systems for managing courses, users, \\ 
            content, and assessments. \\
            - Allow administrators to manage user roles and access control. \\
            - Offer tools for tracking student progress and performance. \\
        \end{tabular} \\ \bottomrule
    \end{tabular}
   \end{adjustbox}
    \caption{Comparison between Blackboard and Madrasty}
    \label{tab:comparison}
\end{table}


\subsection{Conclusion}
Blackboard effectively enhances the learning experience by
enhancing communication between students and teachers and
providing easy access to resources. The reasons for Blackboard, the
problems it solves, and some of its advantages and disadvantages
are explained. A survey was also conducted to identify the problems
some people face and suggest ways to improve them. Despite the
challenges some users may encounter, the many benefits it offers
make it an excellent choice for modern education.

\newpage 

\begin{center}
\vspace*{10cm} 
\section{Chapter 3}
\end{center}

\newpage

This chapter introduces Blackboard design and its importance in improving the
user experience. It explores design principles that enhance navigation, access, and engagement, and examines how design choices contribute to creating an interactive learning environment. Additionally, it presents the interfaces, application data models, databases, and programming languages used to develop the Blackboard application. 

\subsection{Interfaces}

\begin{figure}[h]
    \centering
    \begin{minipage}{0.30\textwidth} 
        \centering
        \includegraphics[width=\linewidth]{Figure4.jpg} 
        \caption{Login interface}
        \label{fig:figure4}
    \end{minipage}\hfill
    \begin{minipage}{0.30\textwidth} 
        \centering
        \includegraphics[width=\linewidth]{Figure5.jpg} 
        \caption{Login loading interface}
        \label{fig:figure5}
    \end{minipage} \hfill 
    \begin{minipage}{0.30\textwidth} 
        \centering
        \includegraphics[width=\linewidth]{Figure6.jpg} 
        \caption{Login to the university website interface}
        \label{fig:figure6}
    \end{minipage}
\end{figure}
\vspace{0.5cm} 


When you enter the application, the interface shown in Figure \ref{fig:figure4} appears, which is
the Blackboard application login screen. Then, the loading interface appears, as shown in Figure \ref{fig:figure5}, after choosing a specific university, in this case Umm Al-Qura University has been chosen. Then, an interface appears to log in to the university website so that it can be linked to the Blackboard account, as shown in Figure \ref{fig:figure6}. The interface includes entering the email and password, then clicking on the word "Login". The application then verifies the validity of the data to ensure a secure registration. \newline

\newpage
\begin{figure}[h]
    \centering
    \begin{minipage}{0.32\textwidth} 
        \centering
        \includegraphics[width=\linewidth]{Figure7.jpg} 
        \caption{Activity Stream interface}
        \label{fig:activity_stream_interface}
    \end{minipage}\hfill
    \begin{minipage}{0.32\textwidth} 
        \centering
        \includegraphics[width=\linewidth]{Figure8.jpg} 
        \caption{Courses interface}
        \label{fig:courses_interface}
    \end{minipage} \hfill 
    \begin{minipage}{0.32\textwidth} 
        \centering
        \includegraphics[width=\linewidth]{Figure9.jpg} 
        \caption{Grades interface}
        \label{fig:grades_interface}
    \end{minipage}
\end{figure}

\noindent
\textbf{Activity Stream:} Displays important activities and updates related to courses, with
its interface shown in (Figure \ref{fig:activity_stream_interface}). The interface has two main sections:
\begin{enumerate}
    \item  The first section is Important: The "Important" section displays important alerts such as links or information related to courses.
    \item  The second section is Upcoming: The "Upcoming Assignments" section
displays information about future assignments.
\end{enumerate}
\noindent
\textbf{The Courses interface: }it can be accessed through the Blackboard app, as shown in (Figure \ref{fig:courses_interface}), where different courses are displayed. This design helps students stay engaged and interact easily with their courses.
\vspace{0.5cm}
\newline
\noindent
\textbf{Grades interface:} Users can view their course grades, as shown in (Figure \ref{fig:grades_interface}).This interface also allows students to easily track their academic progress and check their grades across different courses.

\newpage

\begin{figure}[h!]
    \centering
    \begin{minipage}[b]{0.30\textwidth}
        \includegraphics[width=\textwidth]{Figure10.jpg}
        \caption{Overview of the More Interface.}
        \label{fig:more_interface}
    \end{minipage}
    \hfill
    \begin{minipage}[b]{0.65\textwidth}
        \includegraphics[width=\textwidth]{Figure11.png}
        \caption{Umm Al-Qura University Website Interface.}
        \label{fig:uqu_website_interface}
    \end{minipage}
\end{figure}


The interface shown in Figure \ref{fig:more_interface} displays the registered student's name, and the "More" menu includes various options such as organization, settings, Help Center, About, and sign out.\newline

The interface shown in Figure \ref{fig:uqu_website_interface} represents the Umm Al-Qura University website. It is designed to facilitate access to services and information provided to students, employees, and visitors.

\newpage

\subsection{Data Models}

\subsubsection{Use Case}

\begin{figure}[h!]
    \centering
    \includegraphics[width=1\textwidth]{Figure12.png} 
    \caption{Use Case.}
    \label{fig:use_case}
\end{figure}

The Blackboard use case model illustrates user roles and actions, as shown in Figure \ref{fig:use_case}. This model helps system designers and developers gain a better understanding of user requirements and interactions within the system.
 \newpage
 
\subsubsection{Activity Diagram}

\begin{figure}[h!]
    \centering
    \includegraphics[width=0.55\textwidth]{Figure13.jpg} % Replace with your image file
    \caption{Activity Diagram.}
    \label{fig:activity_diagram}
\end{figure}

The activity diagram is used to track and document all activities performed by
users within the Blackboard application, whether they are students or professors. This model, shown in Figure \ref{fig:activity_diagram}, helps collect information about users’ interactions
with the academic system and includes several main elements.

\newpage

\subsection{Databases}

The Blackboard database is a vital element for organizing and managing information on educational activities. It features a structure dependent on various data models, including users, courses, assignments, and grades, which are organized in tables. The database stores different types of data, such as user information (name and e-mail), class details (title and description), assignments (due dates, instructions), scores, and notes. Relationships between tables connect the data, such as linking the user table with course schedules and submission timelines.

Common database systems used in the Blackboard application include:
\begin{itemize}
    \item \textbf{Microsoft SQL Server:} A popular database in educational systems, providing good performance and scalability.
    \item \textbf{MySQL:} An open-source database known for its flexibility and ease of management, often used in educational applications.
\end{itemize}

\subsection{Programming Languages}

The Blackboard application relies on a variety of programming languages and technologies suited for different tasks:
\begin{itemize}
    \item \textbf{JavaScript:} It is used in the front-end to interact with users and develop dynamic
interfaces. Libraries and frameworks such as React or Angular are also used to
improve the user experience. 
    \item \textbf{PHP:} It may be used in some legacy applications or in specific cases such as
systems that need compatibility with certain applications or services. 
    \item \textbf{SQL:} It is used to query and interact with databases, whether relational such as
MySQL or Oracle.
\end{itemize}

\subsection{Conclusion}

In conclusion, this chapter on the design of the Blackboard application plays a
crucial role in enhancing user experience and facilitating effective learning. Its
intuitive interface and responsive features promote engagement and accessibility for
both students and educators. The design addresses various educational needs,
ensuring that resources are easily navigable.

\newpage

\section{Final Conclusion}

Blackboard has emerged as a powerful tool for education, transforming the
way knowledge is shared and acquired. By providing a comprehensive suite of
features, Blackboard has addressed many of the challenges faced by traditional
education systems. As technology continues to evolve, Blackboard is likely to
play an even more significant role in shaping the future of education.


\section{References}

\begin{itemize}
    \item Blackboard Inc. (n.d.). Blackboard. Retrieved from \url{https://www.blackboard.com/}
    \item Educational Technology Solutions. (n.d.). Blackboard: A comprehensive guide. Retrieved from \url{https://www.blackboard.com/}
    \item University of Michigan. (n.d.). Blackboard help. Retrieved from \url{https://dataguide.umflint.edu/services/view/13}
    \item The Chronicle of Higher Education. (2023). Blackboard's new CEO aims to reimagine higher education. Retrieved from \url{https://www.chronicle.com/}
    \item Blackboard Help. (n.d.). Blackboard app feature guide. Retrieved from \url{https://help.blackboard.com/ar-sa/Blackboard_App/Feature_Guide}
    \item Belaraby Apps. (n.d.). منصة مدرستي السعودية. Retrieved from \url{https://www.belarabyapps.com/}
    \item Educational Wave. (n.d.). Pros and cons of Blackboard. Retrieved from \url{https://ar.educationalwave.com/pros-and-cons-of-blackboard/}
    \item Aspiring Youths. (n.d.). Advantages and disadvantages of Blackboard learning system. Retrieved from \url{https://aspiringyouths.com/advantages-disadvantages/blackboard-learning-system/}
    \item For9a. (n.d.). ماذا تعرف عن البلاك بورد. Retrieved from \url{https://www.for9a.com/learn/}
    \item Blackboard FAQ. (n.d.). What database does Blackboard use? Retrieved from \url{https://blackboard-faq.com/what-database-does-blackboard-use}
    \item Lane, C. A., & Yamashiro, G. (2004). The Blackboard Learning System: The be-all and end-all in educational instruction?. Retrieved from \url{https://www.researchgate.net/publication/240580066_The_Blackboard_Learning_System_The_Be_All_and_End_All_in_Educational_Instruction}
    \item Blackboard Inc. (n.d.). Databases in Blackboard Learn. Blackboard Help. Retrieved from \url{https://help.blackboard.com/ar-sa/Learn/Administrator/Hosting/Databases}
\end{itemize}

\end{document}